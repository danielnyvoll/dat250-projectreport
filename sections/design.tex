\section{Design}
\label{sec:design}

The architecture of our voting application is built upon a microservice framework, utilizing Spring Boot for the backend services and React for the frontend user interface. This section outlines the high-level architecture and implementation details of the application.

\subsection{Client-Side (Frontend)}
The frontend is implemented using React, which provides:
\begin{itemize}
  \item A dynamic and responsive user interface for end-users.
  \item Modular components that communicate with the backend via RESTful APIs.
\end{itemize}

\subsection{Server-Side (Backend)}
Powered by Spring Boot, the backend is composed of:
\begin{itemize}
  \item Microservices that handle specific business capabilities, running independently and communicating through RabbitMQ for messaging.
  \item A layered architecture comprising Controllers, Services, Repositories, and Models.
\end{itemize}

\subsubsection{Application Layers}
The backend is divided into several layers, each responsible for distinct roles within the application:
\begin{itemize}
  \item \textbf{Controller Layer}: Manages HTTP requests and responses.
  \item \textbf{Service Layer}: Orchestrates business logic and data flow.
  \item \textbf{Repository Layer}: Provides data access to the database.
  \item \textbf{Model Layer}: Represents the application's domain model.
\end{itemize}

\subsubsection{Data Management and Storage}
Data persistence is achieved through:
\begin{itemize}
  \item JPA with Hibernate for object-relational mapping and database interactions.
  \item An H2 in-memory database for development, ensuring speed and ease of configuration.
\end{itemize}

\subsubsection{Security and Authentication}
Firebase Authentication is integrated to provide:
\begin{itemize}
  \item Secure authentication mechanisms including social logins and OAuth2 flows.
  \item An abstraction layer for security concerns, allowing focus on core functionalities.
\end{itemize}

\subsubsection{Real-Time Messaging and Updates}
The integration of Dweet.io facilitates:
\begin{itemize}
  \item Real-time messaging capabilities, particularly beneficial for IoT scenarios.
  \item Immediate data update dissemination to clients.
\end{itemize}

\subsection{Conclusion}
The application leverages the strengths of its chosen technologies to provide a robust, scalable, and performant system. The architecture ensures a clean separation of concerns, promoting maintainability and flexibility to cater to the evolving requirements of the application.


About 4 pages on:

\begin{enumerate}

\item An architectural overview of the application that has been implemented
\item High-level design, domain model, … (App assignment A)
\item May involve selected models from Chaps. 5 of the IoT and cloud books


\end{enumerate}

The example below shows how you may include code. There are similar
styles for many other langages - in case you do not use Java in your
project. You can wrap the listing into a figure in case you need to
refer to it. How to create a figure was shown in Section~\ref{sec:technology}.

\lstinputlisting[language=java]{code/BoksVolum.java}
